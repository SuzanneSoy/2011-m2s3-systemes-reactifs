\documentclass[a4paper,french,9pt]{article}
\usepackage[reset]{geometry}
\geometry{a4paper, top=2cm, bottom=2.0cm, left=4.4cm, right=4.47cm}
\usepackage[frenchb]{babel}
\usepackage[utf8]{inputenc}
\usepackage[T1]{fontenc}
\usepackage{amsmath}
\usepackage{amssymb}
\usepackage{centernot}
\usepackage{multirow}
\usepackage{tikz}
\usetikzlibrary{positioning,calc,chains,intersections}
\def\P{\mathcal{P}}
\def\GUi{G \cup \{i\}}
\def\nottransition#1{\stackrel{#1}{\centernot\longrightarrow}}
\def\transition#1{\stackrel{#1}{\longrightarrow}}
\def\Transition#1{\stackrel{#1}{\Longrightarrow}}
\def\forte{\sim}
\def\observationnelle{\approx}
\def\conf{\ \text{conf}\ }
\DeclareMathOperator{\Tr}{Tr}
\DeclareMathOperator{\Acc}{Acc}
\DeclareMathOperator{\Ref}{Ref}
\def\si{\quad\text{si}\quad}
\let\simule\gtrsim
\let\estsimulepar\gtrsim
\begin{document}

\section{Définitions}

\begin{description}
\item[LTS] (Labeled Transition System = Système de transitions étiquetées) : $$P = \langle\P,A,\rightarrow,P\rangle$$
  \begin{description}
  \item[$\P$]: États. $\P = \{P, Q, R, \dots\}$.
  \item[$A$]: Actions (étiquettes que l'on mettra sur les transitions). $A \subseteq \GUi$. $G$ est l'ensemble des actions observables. $i$
    est l'action interne que le système peut faire de lui-même.
  \item[$\rightarrow$]: Transitions. $\rightarrow \in \P\times A\times\P$. $P\transition{a}Q$~: le système se tranforme, passe de $P$ à $Q$
    en exécutant $a$.
  \item[$P$]: Nom du LTS, et aussi son premier état (donc $P \in \P$).
  \end{description}
\item[Trace]: Ensemble des listes d'actions que l'on peut exécuter. La trace vide est notée $\varepsilon$.
  \begin{center}
    \begin{tikzpicture}[node distance=5mm]
      \node (p) {P};
      \node[above right= of p] (q) {Q};
      \node[below right= of p] (r) {R};
      \node[right= of q] (stop1) {Stop};
      \node[right= of r] (stop2) {Stop};
      \draw[->] (p) -- node[above,near start] {a} (q);
      \draw[->] (p) -- node[below,near start] {a} (r);
      \draw[->] (q) -- node[above] {b} (stop1);
      \draw[->] (r) -- node[above] {c} (stop2);
    \end{tikzpicture}
  \end{center}
  $$\Tr(P) = \{\varepsilon, a, ab, ac\}$$
\item[Chemin]: Si $P\transition{a_1}P_1\transition{a_2}\cdots\transition{a_n}P_n$ où $a_i \in \GUi$, on note $P\transition{t}P_n$ avec $t =
  a_1; a_2; \dots; a_n$. $(a_3,a_4)$ est un sous-chemin de $P$.
\item[Exécution] Une exécution partielle de $P$ est un chemin de $P$ partant de l'état initial $P$. Une exécution complète de $P$ est une q
  exécution partielle de $P$ que l'on ne peut pas compléter (j'ai pas compris là…).
\item[Chemin observable] On note $P_1\Transition{a}P_2$ avec $a \neq i$ si $P_1\transition{i^*}Q\transition{a}R\transition{i^*}P_2$. $i^*$ q
  signifie 0 ou plusieurs exécutions de l'action interne.
  
  Si $P\Transition{a_1}P_1\Transition{a_2}\cdots\Transition{a_n}P_n$ où $a_i \neq i$, on note $P\Transition{t}P_n$ avec $t = a_1; a_2;
  \dots; a_n$.
  
  $\hat{t}$ est le chemin obtenu en enlevant toutes les actions internes de $t$. Si $P\transition{t}Q$ alors $P\Transition{\hat{t}}Q$.
\item[Trace observable] Une trace observable de $P$ est une exécution partielle observable.
\end{description}

\section{LOTOS}

\def\fakerowspace{&&&&\\}
\hspace{-2.75cm}
\begin{tabular}{|lllcl|}
  \hline
  %
  \fakerowspace
  \multirow{1}{2cm}{Action prefix} & \multirow{1}{*}{$;:A\times P \rightarrow P$}
  & $\text{ACT}$ & $\frac{\vphantom{P\transition{a}P'}}{a;P\transition{a}P}$ & $a \in \GUi$ \\
  \fakerowspace
  \hline
  %
  \fakerowspace
  \multirow{3}{2cm}{Choice} & \multirow{3}{*}{$[]:P\times P \rightarrow P$}
    & $\text{CHOICE}_1$ & $\frac{P\transition{a}P'}{P[]Q\transition{a}P'}$ & $a \in \GUi$ \\
  \fakerowspace
  & & $\text{CHOICE}_2$ & $\frac{Q\transition{a}Q'}{P[]Q\transition{a}Q'}$ & $a \in \GUi$ \\
  \fakerowspace
  \hline
  %
  \fakerowspace
  \multirow{3}{2cm}{Composition parallèle indépendante} & \multirow{3}{*}{$|||:P\times P \rightarrow P$}
    & $\text{COM}_1$ & $\frac{P\transition{a}P'}{P|||Q\transition{a}P'|||Q}$ & $a \in \GUi$ \\
  \fakerowspace
  & & $\text{COM}_2$ & $\frac{Q\transition{a}Q'}{P|||Q\transition{a}P|||Q'}$ & $a \in \GUi$ \\
  \fakerowspace
  \hline
  %
  \fakerowspace
  \multirow{5}{2cm}[-2mm]{Synchro totale} & \multirow{5}{*}[-2mm]{$||:P\times P \rightarrow P$}
    & $\text{COMP}_1$    & $\frac{P\transition{a}P'\;\wedge\; Q\transition{a}Q'}{P||Q\transition{a}P'||Q'}$ & $a \in \GUi$ \\
  \fakerowspace
  & & $\text{COMP}_{i1}$ & $\frac{P\transition{i}P'}{P||Q\transition{i}P'||Q}$ & \\
  \fakerowspace
  & & $\text{COMP}_{i2}$ & $\frac{Q\transition{i}Q'}{P||Q\transition{i}P||Q'}$ & \\
  \fakerowspace
  \hline
  %
  \fakerowspace
  \multirow{5}{2cm}[-2mm]{Synchro sur un ensemble $S \subseteq$ de portes visibles} & \multirow{5}{*}[-2mm]{$|[S]|:P\times\mathcal{P}(G)\times P \rightarrow P$}
    & $\text{SYNC}_1$ & $\frac{P\transition{a}P'}{P|[S]|Q\transition{a}P'|[S]|Q}$ & $a \in G, a \not\in S, a \neq i$ \\
  \fakerowspace
  & & $\text{SYNC}_2$ & $\frac{P\transition{a}P'}{P|[S]|Q\transition{a}P'|[S]|Q}$ & $a \in G, a \not\in S, a \neq i$ \\
  \fakerowspace
  & & $\text{SYNC}_3$ & $\frac{P\transition{a}P'\;\wedge\; Q\transition{a}Q'}{P|[S]|Q\transition{a}P'|[S]|Q'}$ & $a \in S, a \neq i$ \\
  \fakerowspace
  \hline
  %
  \fakerowspace
  \multirow{3}{2cm}{Rendre une action invisible} & \multirow{3}{*}{$\text{\small hide}\dots\text{\small{}in}:\mathcal{P}(G)\times P \rightarrow P$}
    & $\text{HIDE}_1$ & $\frac{P\transition{a}P'}{(\text{\small hide }a\text{\small{} in }P)\transition{i}(\text{\small hide }a\text{\small{} in }P')}$ & $a \in G, a \neq i$ \\
  \fakerowspace
  & & $\text{HIDE}_2$ & $\frac{P\transition{i}P'}{(\text{\small hide }a\text{\small{} in }P)\transition{i}(\text{\small hide }a\text{\small{} in }P')}$ & $a \in G, a \neq i$ \\
  \fakerowspace
  \hline
  %
  \fakerowspace
  \multirow{1}{2cm}{Renommage} & \multirow{1}{*}{$P[\dots]:P\times G \times G \rightarrow P$}
  & $\text{REN}$ & $\frac{P[a]\transition{a}P'[a]}{P[p]\transition{b}P'[b]}$ & $\begin{array}{ll}a \in G,& a \neq i\\b \in G,& b \neq i\end{array}$ \\
  \fakerowspace
  \hline
  %
\end{tabular}

\clearpage
\section{Comparaison de LTS}

\subsection{Équivalence forte $\forte$}

Def. $P \forte Q\si \forall a \in \GUi,\quad \left(P \transition{a} P'\right) \Rightarrow \left(\exists\, Q \transition{a} Q' \;\wedge\; P' \forte Q'\right)$, et vice versa de $Q$ vers $P$~: $\left(Q \transition{a} Q'\right) \Rightarrow \left(\exists\, P \transition{a} P' \;\wedge\; P' \forte Q'\right)$

En français ça donne~: quand $P$ fait une action, $Q$ sait la faire aussi, et vice versa, donc $Q$ est capable de simuler $P$, et $P$ est
capable de simuler $Q$. On appelle ça la bisimulation.

Si $P$ évolue en interne, $Q$ \emph{doit} évoluer en interne.

L'équivalence forte permet de tester si deux processus ont le même degré d'indéterminisme externe.

\subsection{Équivalence observationnelle $\observationnelle$}

Def. $P \observationnelle Q\si \forall a \in \GUi,\quad \left(P\transition{a}P'\right) \Rightarrow \left(\exists\,
  Q\Transition{\hat{a}}Q' \;\wedge\; P' \observationnelle Q'\right)$ et vice vesa de $Q$ vers $P$~: $\left(Q\transition{a}Q'\right)
\Rightarrow \left(\exists\, P\Transition{\hat{a}}P' \;\wedge\; P' \observationnelle Q'\right)$.

Si $P$ évolue en interne, $Q$ \emph{peut ne pas} évoluer en interne.

Si $P$ fait une action, $Q$ peut faire beaucoup de $\transition{i}$, mais doit faire l'action.

L'équivalence observationnelle ne détecte pas tous les livelock. $P$ et $Q$ sont observationnellement équivalents alors que Q peut refuser
$b$ indéfiniment et toujours faire i.

$$P\transition{a}\transition{b} \quad\observationnelle\quad Q\transition{a}\tikz[remember picture]{\node[coordinate] (ab) {};}\transition{b}$$
\begin{tikzpicture}[remember picture, overlay]
  \node[coordinate,yshift=1mm] (abs) at (ab) {};
  \draw (abs) edge[->,out=45,in=135,looseness=1,loop,distance=0.8cm,shorten <=1mm, shorten >=1mm] node[fill=white,inner sep=1.5pt] {\scriptsize $i$} (abs);
\end{tikzpicture}

\subsection{Congruence opérationnelle $=$}

Def. $P=Q \si \forall a \in \GUi,\quad \left(P\transition{a}P'\right) \Rightarrow \left(\exists\, Q\Transition{a}Q'
  \;\wedge\; P' \observationnelle Q'\right)$ et vice versa de $Q$ vers $P$~: $\left(Q\transition{a}Q'\right) \Rightarrow \left(\exists\,
  P\Transition{a}P' \;\wedge\; P' \observationnelle Q'\right)$.

La différence avec l'équivalence opérationnelle est le $Q\Transition{a}Q'$ au lieu de $Q\Transition{\hat{a}}Q'$. Il est à noter qu'une fois
qu'on a vérifié que la première transition n'était pas l'action invisible, on utilise l'équivalence observationnelle.

Cela a pour but d'avoir $\transition{a}\transition{b}\neq\transition{i}\transition{a}\transition{b}$, mais
$\transition{a}\transition{b}=\transition{a}\transition{i}\transition{b}$.

Cependant, comme pour l'équivalence observationnelle, on ne détecte pas tous les livelock (même exemple que ci-dessus).


\subsection{Simulation}

Def. $P \simule Q \si \forall a \in \GUi,\quad \left(Q\transition{a}Q'\right) \Rightarrow \left(\exists\, P\Transition{a}P' \;\wedge\; P'
  \simule Q'\right)$. On dit \og $P$ simule $Q$\fg.

Attention, la bisimulation $P \forte Q$ n'est pas $\left(P \simule Q \;\wedge\; P \estsimulepar Q\right)$, car dans simul, à la fin on a $P'
\simule Q'$, alors que dans la bisimulation on a $P' \forte Q'$. La bisimulation $P \forte Q$ est donc une condition plus forte que la
simulation mutuelle de P et Q $\left(P \simule Q \;\wedge\; P \estsimulepar Q\right)$.

\subsection{Force des comparaisons}

$$ P \forte Q \Rightarrow P = Q \Rightarrow P \observationnelle Q $$

\clearpage
\subsection{Conformité $\conf$}

Les comparaisons des sections précédentes ne détectent pas tous les livelock, et nécessitent une description complète de la spécification,
car ces relations sont symmétriques. On cherche donc une relation assymétrique de conformité, $P\conf S$ qui nous permettra de dire que $P$
est une implantation correcte de $S$.

\begin{description}
\item[must] On écrit $S\text{ must }A\text{ after }t$ pour dire que $S$ doit accepter les actions de l'ensemble $A$ après le chemin $t$.

  Def. $Q\text{ must }A\text{ after }t \si \forall Q \Transition{t} Q',\quad \forall a \in A, \quad Q'\transition{a}Q"$.

  $\Acc(P,t)$ est l'ensemble des actions que $P$ doit accepter après $t$.
  
  Def. $\Acc(P,t) = \left\{X \subseteq G | \exists P', P \Transition{t} P', \forall x \in X, P' \transition{x} P"\right\} $
\item[may refuse] On écrit $P\text{ may refuse }A\text{ after }t$ pour dire que $P$ peut refuser une des actions de $A$\footnote{\dots ou
    toutes les actions de $A$, selon d'autres définitions, mais pas ici.} après le chemin $t$.

  Def. $Q\text{ may refuse }A\text{ after }t \si \forall Q \Transition{t} Q',\quad \exists a \in A, \quad Q'\nottransition{a}$.
  
  $\Ref(P,t)$ est l'ensemble des actions que $P$ peut (doit?) refuser après $t$.
  
  Def. $\Ref(P,t) = \left\{X \subseteq G | \exists P', P \Transition{t} P', \forall x \in X, P' \nottransition{x}\right\} $
\end{description}

\def\myfnote{\footnote{Je ne sais pas trop si c'est $\forall A \subseteq G$ ou $\forall A \subseteq \GUi$.}}%
$$
P \conf S\si\forall A \subseteq G\myfnote,\quad S\text{ must }A\text{ after }t \Rightarrow P\text{ must }A\text{ after }t
$$

En prennant la contraposée, on obtient un test de conformité (équivalent à la définition ci-dessus)~:

$$
P \conf S\si\forall t \in \Tr(S),\quad P\text{ may refuse }A\text{ after }t \Rightarrow S\text{ may refuse }A\text{ after }t
$$

Autrement dit,

$$
P \conf S\si\forall t \in \Tr(S),\quad \Ref(P,t) \subseteq \Ref(S,t)
$$

Encore une formulation~: $\forall p, p \in \text{vivacité}(S) \Rightarrow p \in \text{vivacité}(P)$, c'est-à-dire $P$ préserve les
propriétés de vivacité de $S$, c'est-à-dire $S\text{ doit accepter }A\text{ après }t \Rightarrow P\text{ doit accepter }A\text{ après }t$,
c'est-à-dire $P \conf S$ (P conforme à S).

Par rapport aux autres relations (attention, ce sont des implications, pas forcément vraies dans l'autre sens)~:
\begin{gather*}
  P = Q \Rightarrow P \conf Q \;\wedge\; Q \conf P\\
  P = Q \Rightarrow \Tr(P) = \Tr(Q)\\
\end{gather*}

\subsection{En français…}

$P\conf Q$ signifie~:
\begin{itemize}
\item P est plus déterministe que Q
\item P se bloque moins souvent que Q
\item P refuse moins souvent que Q
\end{itemize}

\section{Raffine}

\begin{gather*}
  S_1\text{ raffine } S_0\si \forall I \in \text{ Implémentations},\quad \left(I \conf S_1 \Rightarrow I \conf S_0\right)\\
  \text{red} = \conf \cap \subseteq_{\Tr}\\
  \text{ext} = \conf \cap \supseteq_{\Tr}
\end{gather*}

\begin{figure}
  \centering
  \begin{tikzpicture}[scale=0.5]
    \draw (0,0) circle (6);
    \node at (0,-4) {$\conf$};
    \draw (-1.5,1) circle (2);
    \node at (-2.5,1) {red};
    \draw (1.5,1) circle (2);
    \node at (2.5,1) {ext};
    \draw (1.5,1) circle (3);
    \draw (1.5,-1.5) -- (6,-5);
    \node[anchor=west] at (6,-5) {confrest = raffinement};
    \draw (0,1) -- (6,-3);
    \node[anchor=west] at (6,-3) {te = équivalence de test};
  \end{tikzpicture}
\end{figure}

\section{Autres notes}

\begin{description}
\item[Vivacité] L'airbag \emph{doit} se déclencher en cas de choc.
\item[Sûreté] Les portes \emph{ne doivent pas} s'ouvrir si la cabine d'ascenceur n'est pas là.
\end{description}

Principe d'\emph{équité}~: «Une action qui se propose infiniment souvent finira par être acceptée».

\section{Exemples de LTS}

\begin{figure}[h]
  \centering
  \begin{tikzpicture}[node distance=5mm]
    \node (b0) {$B_0$};
    \node[below=of b0] (b1) {$B_1$};
    \node[below=of b1] (b2) {$B_2$};
    \node[below=of b2] (b1p) {$B_1'$};
    \node[below=of b1p] (b2p) {$B_2'$};
    \node[at=(b0), xshift=1.5cm, coordinate] (b0r) {};
    
    \draw[->] (b0) -- node[right] {$\text{inp}_1$} (b1);
    \draw[->] (b1) -- node[right] {$\text{inp}_2$} (b2);
    \draw[->] (b2) -- node[right] {$\text{outp}_1$} (b1p);
    \draw[->] (b1p) -- node[right] {$\text{inp}_1$} (b2p);
    
    \draw[->] (b1) edge[bend left] node[left] {$\text{outp}_1$} (b0);
    \draw[->] (b2p) edge[bend left] node[left] {$\text{outp}_2$} (b1);

    \draw[->] (b1p) -| node[below] {$\text{outp}_2$} (b0r) -- (b0);
    % \draw[->] (p) -- node[right,near start] {a} (r);
    % \draw[->] (q) -- node[above] {b} (stop1);
    % \draw[->] (r) -- node[above] {c} (stop2);
  \end{tikzpicture}
  \caption{File d'attente à deux places}
\end{figure}

\end{document}
